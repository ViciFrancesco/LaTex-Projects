\documentclass[8pt]{extarticle}
\usepackage[a4paper,headsep=10pt, top=30pt, bottom=30pt, left=30pt, right=30pt, footskip=10pt]{geometry}
\usepackage{listings}
\usepackage{setspace}			%used for \begin{spacing}\end{spacing}-> vertical space setting
\usepackage{amsfonts} 		%used for \mathbb{} -> numerical sets
\usepackage{amsmath}			%used for \text{} inside math equations
\usepackage{amssymb}			%more math symbols
\usepackage{enumitem}
\usepackage{bm}					%used for bold math expressions
\usepackage{calc}				%used for algebric operations inside \newenvironment
\usepackage{graphicx}			%used for custom limits
\usepackage{fancyhdr}			%used for head and foot settings
%\usepackage{draftwatermark}		%used for watermark
%
%
%%%watermark settings
%\SetWatermarkText{Vici Francesco}
%\SetWatermarkScale{2}
%\SetWatermarkAngle{60}
%%end watermark settings
\usepackage{relsize}


\fancypagestyle{plain}
{
\fancyhead{}\fancyfoot{}
\fancyhead[C]{\vspace{0pt}   Formulario Analisi II}
\fancyfoot[R]{Vici Francesco}
\fancyfoot[C]{\thepage}
}
\pagestyle{plain}



\newenvironment{formulario}
{
\setlength{\columnsep}{3em}
\twocolumn
\lstset{tabsize=3}
\begin{spacing}{1}
\begin{flushleft}
}{
\end{flushleft}
\end{spacing}
}



\newenvironment{tcenter}{
  \par
  \centering
  \setlength{\parskip}{0pt} % Rimuovi spaziatura verticale
  \noindent
}{
  \par
}



\newenvironment{descr}[1]
{
\setlist[description,1]{leftmargin=2em + (2em * #1),labelindent=0em + (2em * #1)}
\begin{description}[topsep=0pt,itemsep=0pt,partopsep=0pt, parsep=0pt]
}{
\end{description}
}



\newenvironment{myParagraph}[1]
{
\begin{tcenter}
\textbf{#1}
\end{tcenter}
}{
\myRule
}



\newcommand{\norma}[1]{\lvert\lvert #1\rvert\rvert}
\newcommand{\modulo}[1]{\lvert #1\rvert}
\newcommand{\module}[1]{\lvert #1\rvert}
\newcommand{\R}{\mathbb{R}}
\newcommand{\N}{\mathbb{N}}
\newcommand{\myRule}{\rule{250pt}{0.1pt}}
\newcommand{\Lim}[1]{\raisebox{0.5ex}{\scalebox{0.8}{$\displaystyle \lim_{#1}\;$}}}
\newcommand{\scalare}{\mathbin{\vcenter{\hbox{\scalebox{.6}{\;$\bullet$\;}}}}}
\newcommand{\bo}[1]{\textbf{#1}}
\newcommand{\ra}{\rightarrow}
\newcommand{\la}{\lefttarrow}
\newcommand{\sse}{\leftrightarrow}
\newcommand{\SSE}{\Longleftrightarrow}


\begin{document}

	\begin{formulario}

%%PARAGRAFO 1
%		\begin{tcenter}
%\textbf{FUNZIONI A PIÙ VARIABILI}
%		\end{tcenter}
%Si dice funzione a più variabili una funzione del tipo: 
%		\begin{tcenter}
%$f:\R^n\to\R^m$
%		\end{tcenter}
%Inoltre possiamo definire "scalare" una funzione del tipo $f:\R^n\to\R$ mentre viene definita "vettoriale" una funzione del tipo $f:\R^n\to\R^m$.
%\myRule

%PARAGRAFO 2
		\begin{tcenter}
\textbf{FUNZIONI A DUE VARIABILI SCALARI}
		\end{tcenter}
È una funzione del tipo $f:\R^2\to\R$, chiamo le variabili $x,y$. \\
		\begin{descr}{0}
			\item[Dominio:] (funzioni più comuni)\\	 
			\begin{descr}{0}
				\item[Frazioni] $\rightarrow$ denominatore $\neq 0$;
				\item[	Radici Pari] $\rightarrow$ argomento $\geq 0$;
				\item[	Logaritmo] $\rightarrow$ argomento $>0$;
				\item[	Funzioni continue] $\rightarrow$ somma, prodotto e composizione di funzioni continue sono a loro volta continue.
			\end{descr} 
%\item[Dominio] $\rightarrow$ In generale, dati $f:D\to\R$ con $D\subset\R^2$, $D$ viene detto dominio.
\item[Segno] $\rightarrow$ Studio dove $f:(x,y)>0,\text{ }=0 \text{ oppure }<0$. Si tratta quindi di determinare uno di questi sottoinsiemi (spesso è utile rappresentarlo graficamente). 
%		\end{descr}
%Al fine di definire la continuità di una funzione è necessario definire i concetti di "distanza" e tutti i concetti che ne derivano e quello di "limite":
		
%		\begin{descr}{1}
%\item[Distanza in $\R^2$] $\rightarrow$ Dati due punti $(x,y) \text{ e } (x_0,y_0)$ la distanza tra loro è data dalla norma a 2, cioè: \\
%			\begin{tcenter}
%$\mathbf{\norma{(x,y)-(x_0,y_0)}=\sqrt{(x-x_0)^2+(y-y_0)^2}}$
%			\end{tcenter}
%\item[Intorno (Circolare)] $\rightarrow$ Dato un punto $(x_0,y_0)$, $I$ viene detto intorno circolare di $(x_0,y_0)$ se:\\
%			\begin{tcenter}
%$\mathbf{I(x_0,y_0)=\{(x,y)\;\big\vert\;\norma{(x-y)-(x_0,y_0)}<r\}}$
%			\end{tcenter}
\item[Interno] $\rightarrow$ Dato un punto $(x_0,y_0)\in A$, tale punto si dice interno ad $A$ se:\\
			\begin{tcenter}
$\mathbf{\exists  I(x_0,y_0)\subset A}$
			\end{tcenter}
\item[Frontiera (o Bordo)] $\rightarrow$ Un punto $(x_0,y_0)$ si dice appartenere alla frontiera (o bordo) di $A$, e si scrive $(x_0,y_0)\in\partial A$ se:\\
			\begin{tcenter}
$\mathbf{\forall I(x_0,y_0)
			\begin{cases}
   	\exists (x,y)\neq(x_0,y_0)\text{ t.c. } (x,y)\in A\cap I(x_0,y_0) \\
   	\exists (z,t)\neq(x_0,y_0)\text{ t.c. } (z,t)\in A^c\cap I(x_0,y_0)
  			\end{cases}}$
			\end{tcenter}
%\item[Punto di Accumulazione] $\rightarrow$ Dato un punto $(x_0,y_0)$, questo si dice punto di accumulazione per $A$ se: \\
%			\begin{tcenter}
%$\forall I(x_0,y_0)\; \exists(x.y)\neq(x_0,y_0) \text{ t.c. } (x,y)\in I(x_0,y_0)\cap A$
%			\end{tcenter}
\item[Insieme Aperto/Chiuso] $\rightarrow$ A è un insieme aperto se coincide con l'insieme dei sui punti interni (cioè $\mathbf{A}$ \textbf{ha tutto il bordo "tratteggiato"}($\bm{</>}$)), mentre A è un insieme se $\partial A\subset A$ (cioè $\mathbf{A}$ \textbf{ha tutto il bordo "continuo"}($\bm{\leq/\geq}$)).

%\item[Chiusura di un Insieme] $\rightarrow$ La chiusura di un insieme $A$ è data dall'insieme stesso unito con la sua frontiera, cioè:
%			\begin{tcenter}
%		$\mathbf{\overline{A}=A\cup\partial A}$
%			\end{tcenter}
		\end{descr}
Inoltre possono anche essere utili altre definizioni secondarie come:
		\begin{descr}{0}
\item[Insieme Limitato] $\rightarrow$ Un insieme $A$ è limitato in $\R^2$ se: 
			\begin{tcenter}
$\mathbf{\exists a,b,c,d \in \R \text{ t.c. } \forall (x,y)\in A: a\leq x\leq b,\; c\leq y\leq d}$\\ 
oppure \\
$\mathbf{\exists c\in\R^+ \text{ t.c. } \forall (x,y)\in A: x^2+y^2\leq c }$  
			\end{tcenter} 
\bo{\underline{PER ES}:} Se ho dubbi faccio limiti agli estremi del dominio e nei punti esclusi dal dominio.
\item[Insieme non Limitato] $\rightarrow$ Un insieme $A$ è non limitato se:
			\begin{tcenter}
$\mathbf{\forall c\; \exists (x,y) \text{ t.c. } \norma{(x,y)}>c}$ 
			\end{tcenter}
\item[Insieme Compatto] $\rightarrow$ Un insieme si dice compatto se è chiuso e limitato.
\item[Insieme Connesso] $\rightarrow$ Un insieme si dice connesso non ai può ricoprire con due insiemi non vuoti, aperti e disgiunti.
		\end{descr} 
\myRule

%PARAGRAFO 3
		\begin{tcenter}
\textbf{LIMITI DI FUNZIONI IN DUE VARIABILI}
		\end{tcenter}
La principale differenza con i limiti di funzioni in una variabile sta nel fatto che in $\R$ esistono solo due possibili "direzioni" per cui una funzione può tendere ad un punto, mentre in $\R^2$ le direzioni sono infinite.
		\begin{descr}{0}
\item[Non Esistenza:] Dato un limite del tipo $\Lim{(x,y)\to(x_0,y_0)}$ è sufficiente trovare due successioni di intervalli tali che $(a_n,b_n)\to (x_0,y_0)$ e per cui il limite abbia valore diverso. Le successioni più comuni sono:
			\begin{descr}{0}
				\item[per]$\bm{(x,y)\to(0,0)}$:
					$(\frac{1}{n},\frac{1}{n}),(\frac{1}{n},0),(0,\frac{1}{n})$ con $n\in\N\to\infty$;
				\item[per]$\bm{(x,y)\to(\infty,\infty)}$:
					$(n,n)$ con $n\in\N\to\infty$.
			\end{descr}
\item[Esistenza:] Dato un limite del tipo $\Lim{(x,y)\to(x_0,y_0)}$ è necessario effettuare un cambio di variabile da $(x,y)$ a $(\rho,\theta)$. Il cambio di variabile sarà allora $(x,y)=(\rho\;\cos\theta+x_0,\rho\;\sin\theta+y_0)$ e il limite diventerà quindi $\Lim{\rho\to \rho_0}$, dove $\rho_0=0$ se il limite è finito, mentre $\rho_0=\infty$ se il limite è infinito. 
%\item [\underline{DEF:} (Per Intorni)] $\rightarrow$ Data $f:D\to\R,D\subset\R^2$ e dato $(x_0,y_0)$ punto di accumulazione per $D$ dico che $\Lim{(x,y)\to(x_0,y_0)}f(x,y)=l$ se $\forall\epsilon >0 \; \exists\delta>0 \;$ t.c. se $(x,y)\in D, \; (x,y)\neq(x_0,y_0)$:
%			\begin{tcenter}
%$\mathbf{\norma{(x,y)-(x_0,y_0)}<\delta \implies \module{f(x,y)-l}<\epsilon}$
%			\end{tcenter}
%\item [\underline{DEF:} (Per Successioni)] $\rightarrow$ Data $f:D\to\R,D\subset\R^2$ e dato $(x_0,y_0)$ punto di accumulazione per $D$ dico che $\Lim{(x,y)\to(x_0,y_0)}f(x,y)=l$ se $\forall (x_n,y_n)\in D\setminus{(x_0,y_0)}, n\in \N$ t.c. se:
%			\begin{tcenter}
%$\mathbf{(x_n,y_n)\to (x_0,y_0) \text{ allora } f(x_n,y_n)\to l}$
%			\end{tcenter}
		\end{descr}
%È anche possibile approcciare il problema sotto il punto di vista delle coordinate polari $(\rho,\Theta)$, questo riduce il problema ad un limite ad una variabile in quanto il parametro $\Theta$ è libero è dobbiamo verificare solo la tendenza di $\rho$.
\myRule

%PARAGRAFO 4
		\begin{tcenter}
\textbf{CONTINUITÀ DI FUNZIONI A DUE VARIABILI}
		\end{tcenter}
%		\begin{descr}{0}
%\item[\underline{DEF:}] $\rightarrow$ Data $f:D\to\R, D\subset\R$ e dato $(x_0,y_0)\in D$ punto di accumulazione per D, si dice che $f$ è continua in $(x_0,y_0)$ se:
%			\begin{tcenter}
%$\mathbf{\exists\Lim{(x,y)\to(x_0,y_0)}f(x,y) \text{ \textbf{ed è pari a} } f(x_0,y_0)}$.
%			\end{tcenter}
%		\end{descr}
%Se la funzione è definita "a tratti" allora la continuità è da verificare nei punti di frontiera comune.
		\begin{descr}{0}
\item[Estensione a tutto $\bm{\R^2}$] $\rightarrow$ Assumendo che la funzione $f(x,y)$ sia continua in tutto $\R$ eccetto che in uno o più punti $P_i$, la sua continuità è estendibile a tutto $\R$ se il limite (in $(\rho,\theta)$) che tende a $P_i$ esiste e ha valore finito.
\item[Teorema di Weierstrass] $\rightarrow$ Data $f:K\to\R$ con $K$ compatto e $f$ continua in $\R$ allora:
			\begin{tcenter}
$\mathbf{f}$ \textbf{ha massimo e minimo assoluti in} $\mathbf{K}$
			\end{tcenter}
		\end{descr}
\myRule

%PARAGRAFO 5
		\begin{tcenter}
\textbf{RAPPRESENTAZIONE GRAFICA DI FUNZIONI A DUE VARIABILI}
		\end{tcenter}
		\begin{descr}{0}
%\item[Superficie Associata] $\rightarrow$ Data $f:D\to\R,\; D\subset\R^2$ definisco la superficie associata ad $f$ come:
% 			\begin{tcenter}
%$\mathbf{S_f=\left\{ (x,y,z)\in\R^3 \bigg\vert 
%\begin{array}{l}
%	\mathbf{(x,y)\in D }\\
%	\mathbf{z=f(x,y)}
%\end{array}
%\right\}} \mathbf{\subset\R^3}$
%			\end{tcenter}
%\item[Insiemi di Livello] $\rightarrow$ Data $f:D\to\R,\; D\subset\R^2$ e dato $c\in\R$, chiamo "insieme di livello di $f$ a livello $c$" l'insieme:
%			\begin{tcenter}
%$\mathbf{U_c:= \left\{ (x,y)\in D \big\vert f(x,y)=c \right\}\subset D}$
%			\end{tcenter}
%\underline{OSS:} 
%			\begin{tcenter}
%$U_c$ può essere: 
%	$\begin{cases}
%\emptyset \\
%\text{insieme di punti isolati} \\
%\text{curva} \\
%\text{regione bidimensionale}
%	\end{cases}$
%			\end{tcenter}
			\item[Insiemi di Livello] $\rightarrow$ Data una funzione $f(x,y)$ per determinarne gli insiemi di livello si deve uguagliare la funzione alla costante $c$ (possibilmente isolando x e y) e vedendo come varia la funzione al variare di $c$. Per disegnarli è sufficiente assegnare a $c$ un valore.
			\item[Crescita della Funzione] $\rightarrow$ Data una funzione $f(x,y)$ per determinarne la crescita è sufficiente porre $=0$ una variabile alla volta, proiettando così la funzione sul piano perpendicolare all'asse della variabile annullata. 
			\begin{descr}{0}
				\item[\underline{PER ES:}] $f(x,y)=\log(x^2+y^2)\implies$ \\
				piano yz: \quad $x=0\longrightarrow f(0,y)=\log(y^2)$;\\
				piano xz: \quad $y=0\longrightarrow f(x,0)=\log(x^2)$.			
			\end{descr}
		\end{descr}
\myRule

%PARAGRAFO 6
		\begin{tcenter}
\textbf{DERIVATE PARZIALI}
		\end{tcenter}
		\begin{descr}{0}
\item[\underline{DEF:}] $\rightarrow$ Sia $(x_0,y_0)$ un punto interno al dominio di $f$. si chiama derivata parziale di $f$ rispetto ad $x$ nel punto $(x_0,y_0)$ , la derivata classica di:
			\begin{tcenter}
$x \to f(x,y_0) \text{ in } x=x_0$
			\end{tcenter}
e si indica con $\frac{\partial f}{\partial x}(x_0,y_0)$. Ricordando la definizione di derivata classica si ha che:		
			\begin{tcenter}
$\mathbf{\frac{\partial f}{\partial x}(x_0,y_0)=\Lim{h\to 0} \frac{f(x_0+h,y_0)-f(x_0,y_0)}{h}}$
			\end{tcenter}
%\underline{\textbf{OSS:}} Si possono usare le stesse regole di derivazione classiche considerando l'altra variabile come una costante. 
%		\end{descr}
%Le derivate parziali si mettono in un vettore chiamato gradiente. Questo, come funzione di $x,y$ è un campo vettoriale:
%		\begin{tcenter}
%$\nabla f(x,y)=(\frac{\partial f}{\partial x}(x,y)\; ,\; \frac{\partial f}{\partial y}(x,y))$
%		\end{tcenter} 
%		\begin{descr}{0}
\item[Derivate Parziali di Ordine Superiore] $\rightarrow$ Per una funzione a due variabili esistono 4 derivate di secondo ordine, infatti ogni derivata prima può essere derivata rispetto ad $x$ o rispetto ad $y$:
			\begin{tcenter}
$D^2\, f(x,y)= 
\begin{pmatrix}
\frac{\partial^2 f}{\partial^2 x} & \frac{\partial^2 f}{\partial y\partial x} \\
\frac{\partial^2 f}{\partial x\partial y} & \frac{\partial^2 f}{\partial^2 y}
\end{pmatrix}$
			\end{tcenter}
Tale matrice viene detta Matrice Hessiana.
			
\item[Teorema di Schwarz] $\rightarrow$ Se f è di classe $C^2$, cioè le derivate seconde esistono e sono continue allora:
				\begin{tcenter}
$\frac{\partial^2 f}{\partial x\partial y}(x,y)=\frac{\partial^2 f}{\partial y\partial x}(x,y)$
				\end{tcenter}
cioè la matrice essiana è simmetrica.
			
		\end{descr}
		\begin{descr}{0}
\item[Differenziabilità] $\rightarrow$ Data $f:A\to\R$ con $A$ aperto di $\R^2$, $\nu\in\R^2$ ($\nu\to0$ significa $\norma{\nu}\to0$), allora $f$ si dice differenziabile  se:
			\begin{tcenter}
$\Lim{\nu \to \underline{0}} \frac{f((x,y)+\nu)-f(x,y)-\nabla f(x,y)\cdot\nu}{\norma{\nu}}=0$ 
			\end{tcenter}
			
\item[Teorema del Differenziale] $\rightarrow$ se $f$ è di classe $C^1 $ (ovvero se le sue derivate prime sono continue e compatte) $ \implies f$ è differenziabile. 
				
 

		\end{descr}
\myRule

%PARAGRAFO 7
		\begin{myParagraph}{PIANO TANGENTE}
Se una funzione è differenziabile, si può approssimare "bene" con un piano. Si chiama Piano Tangente il piano in $\R^3$ di equazione:
			\begin{tcenter}
$z:=f(x_0,y_0)+\nabla f(x_0,y_0)\scalare (x-x_0,y-y_0)$
			\end{tcenter}
è della forma $z=ax+by+c$ e il punto $(x_0,y_0,f(x_0,y_0))$ è il punto di contatto tra il piano e la funzione.
			\begin{descr}{0}
				\item[Piano Tangente Orizzontale] $\rightarrow$ Data $f(x,y)$ il piano tangente è orizzontale solo nei punti in cui $\nabla f(x,y)=(0,0)$. Inserendo i punti in cui il gradiente si annulla nell'equazione del piano si ottiene la quota.
				\item[Piano Parallelo ad una Retta] $\rightarrow$ Data l'equazione di un piano tangente $z$ e l'equazione di una retta in $\R^3$ nella forma $ax=by=cz$ per ottenere i piani paralleli alla retta si deve ricavare il vettore direzione della retta cioè $[\frac{1}{a},\frac{1}{b},\frac{1}{c}]$ e sostituirla al posto di $[x,y,z]$ (non al posto di $[x_0,y_0]$) nel vettore in modo da poter ricavare una possibile variabile $\alpha$ dall'equazione del piano. 
			\end{descr}
		\end{myParagraph}

%PARAGRAFO 8
		\begin{tcenter}
\textbf{DERIVATA DIREZIONALE}
		\end{tcenter}
		
Dato $\nu$ un vettore unitario di $\R^2$ (cioè $\norma{\nu}=1$) la derivata direzionale di $f$ in direzione $\nu$ nel punto $(x_0,y_0)$ è la derivata classica di:
		\begin{tcenter}
$t \to f((x_0,y_0)+t\cdot\nu)$
		\end{tcenter} 
cioè:
		\begin{tcenter}
$\mathbf{\Lim{f\to0} \frac{f((x_0,y_0)+h\cdot\nu)-f(x_0,y_0)}{h}}$
		\end{tcenter}
e si indica con $\frac{\partial f}{\partial \nu}(x_0,y_0)$.
		\begin{descr}{0}
\item[Teorema] $\rightarrow$ Se $f$ è differenziabile, allora:
			\begin{tcenter}
$\frac{\partial f}{\partial\nu}(x,y)=\nabla f(x,y)\scalare\nu$
			\end{tcenter}
		\end{descr}
		\begin{descr}{1}
			\item[\underline{PER ES}:] Data $f(x,y)$ e $(x_0,y_0)$ trovare $\nu$ t.c. $\frac{\partial f}{\partial \nu}(x_0,y_0)$ sia massima/minima. \\ 
				Il modo migliore per svolgerlo è trovare il versore del gradiente e e il suo opposto. Quelli saranno i due $\nu$ che minimizzeranno/massimizzeranno la derivata direzionale in $(x_0,y_0)$. 
			\begin{tcenter}
				$\nu_{MAX}=\frac{\nabla f(x_0,y_0)}{\norma{f(x_0,y_0)}}$ \\
				$\nu_{MIN}=-\frac{\nabla f(x_0,y_0)}{\norma{f(x_0,y_0)}}$
			\end{tcenter}
		\end{descr}
\myRule

%PARAGRAFO 9
		\begin{myParagraph}{MASSIMI E MINIMI (ESTREMI)}
			\begin{descr}{0}
				\item[Individuazione dei Punti Critici] $\rightarrow$ Come nel caso 1D non è sufficiente che $\nabla f(x_0,y_0)=(0,0)$ affinché $(x_0,y_0)$ sia un massimo/minimo. Devo quindi caratterizzare maggiormente la natura di questi punti.
				\begin{descr}{0}
					\item[Condizione Necessaria] $\rightarrow$ se $(x_0,y_0)\in D_f$ (dominio della funzione) è un estremo locale/globale $\implies 		
					\begin{cases}
						\nexists \nabla f(x_0,y_0)\implies \text{PUNTO SINGOLARE}\\
						\nabla f(x_0,y_0)=(0,0)\implies \text{PUNTO CRITICO}\\
						(x_0,y_0)\in \partial D_f\implies \text{PUNTO DI BORDO}
					\end{cases}$
%					\item[\underline{DIM:}] Dato $(x_0,y_0)\in D°_f$ (parte interna di $D_f$)  estremo locale/globale, supponiamo che $\exists\nabla f(x_0,y_0)$. Per assurdo supponiamo che sia $\neq(0,0)$:\\
%					\begin{tcenter}
%					$\implies$ \underline{wlog} $\frac{\partial f}{\partial x}(x_0,y_0)\neq0 \implies$\\
%					$\implies \Lim{h\to0}\frac{f(x_0+h,y_0)-f(x_0,y_0)}{h}=l\neq0, $ \underline{wlog} $ l>0 \implies$\\
%					$\implies$ per $h$ piccolo si ha $\frac{f(x_0+h,y_0)-f(x_0,y_0)}{h}>0\implies$\\
%					\end{tcenter}
%					$\begin{cases}
%					\text{se prendo }h>0 \implies f(x_0+h,y_0)>f(x_0,y_0)\implies\\
%					\text{se prendo }h<0 \implies f(x_0+h,y_0)<f(x_0,y_0)\implies\\
%					\end{cases}$ 
%					$\begin{array}{l}
%						\implies \text{NO MAX LOCALE}\\
%						\implies \text{NO MIN LOCALE}
%					\end{array} \implies$ \underline{Contraddizione}.
					\item[N.B.] $\rightarrow$ se $f:D\to\R$ continua su $D$ compatto allora trovo sicuramente max/min assoluti confrontando punti critici/singolari/bordo.
				\end{descr}
				\item[Caratterizzazione dei Punti Critici] Nel caso 1D si osserva la derivata seconda per capire la natura dei punti critici. Poiché la derivata seconda in 2D corrisponde alla matrice hessiana abbiamo bisogno di ulteriori osservazioni:
				\begin{descr}{0}
					\item[\underline{DEF}:(Preliminare)] $\rightarrow$ Data $M$ matrice $2\times2$ diagonalizzabile (ha 2 autovalori reali $\lambda_1,\lambda_2$) si dice:
					$\begin{array}{l}
					\text{(SEMI) DEF. POS. se }\lambda_1>0,\lambda_2>0 \; (\lambda_1\geq0,\lambda_2\geq0) \\
					\text{(SEMI) DEF. NEG. se } \lambda_1<0,\lambda_2<0 \; (\lambda_1\leq0,\lambda_2\leq0)\\
					\text{INDEFINITA se} \lambda_1>0,\lambda_2<0 \text{ (o viceversa) }.
					\end{array}$
					\item[\underline{PER ES}:] Per caratterizzare la matrice hessiana è anche possibile usare traccia e determinante: 
					$\begin{array}{l}
					\text{(SEMI) DEF. POS. se } Tr(D^2)>0, Det(D^2)>0; \\
					\text{(SEMI) DEF. NEG. se } Tr(D^2)<0, Det(D^2)<0; \\
					\text{INDEFINITA se } Tr(D^2) \text{ e } Det(D^2)\text{ sono discordi}.
					\end{array}$
					\item[Teorema] $\rightarrow$ sia $f\in C^2$ definita su un aperto $A\subseteq \R^2$ si ha:
					\begin{descr}{0}
%						\item[Condizione Necessaria] $\rightarrow$ $(x_0,y_0)$ è un max (min) locale $\implies D^2f(x_0,y_0)$ è Semi Definita Negativa (Positiva).
						\item[Condizione Sufficiente] $\rightarrow$ $(x_0,y_0)$ t.c. $\nabla f(x_0,y_0)=(0,0)$ se $D^2f(x_0,y_0)$ DEF. NEG. (POS.) $\implies (x_0,y_0)$ è max (min locale);  se invece è INDEFINITA $\implies (x_0,y_0)$ è un PUNTO DI SELLA. \\
						\item[\underline{N.B.}:] Negli altri casi devo analizzare le derivate di ordine superiore.
						\item[\underline{OSS:}] $\rightarrow$ $D^2f$ è simmetrica perché $f\in C^2 \implies D^2f$ è diagonalizzabile.
					\end{descr}
					\item[Teorema di Weierstrass] $\rightarrow$ Data $f$ continua definita su $A$ compatto $\implies$ \\
					\begin{tcenter}
						$\exists \max_A f, \exists \min_A f$.
					\end{tcenter}
					\begin{descr}{0}
					\item[\underline{OSS:}] $\rightarrow$ $A$ è limitato e chiuso perché $\partial A\subset A$.
					\end{descr}
					I punti di $\max / \min$ vanno cercati tra:
					\begin{tcenter}
						$\begin{cases}
							\text{Punti Critici Interni }\ra \text{non ci sono} \\
							\nexists\nabla f \ra \text{non ci sono} \\
							\partial A \ra \text{\underline{ci sono}}
						\end{cases}$
					\end{tcenter}
				\end{descr}
			\end{descr}
		\end{myParagraph}


%PARAGRAFO 10
		\begin{myParagraph}{SVILUPPI DI TAYLOR}
Data una funzione $f$ differenziabile in $(x_0,y_0)\in D$ aperto di $\R^2$ allora gli sviluppi di Taylor sono:\\
			\begin{descr}{0}
				\item[Grado 1]:\\
				\begin{descr}{0}
					\item[$f(x,y)=$]$f(x_0,y_0)+\nabla f(x_0,y_0)\scalare(x-x_0,y-y_0)+$ $+\underbrace{(\norma{(x-x_0,y-y_0)})}_\text{Resto}$;
					\item[$f(x+h,y+k)=$]$=f(x_0,y_0)+\nabla f(x_0,y_0)\scalare (h,k)+\underbrace{o(\sqrt{h^2+k^2})}_\text{Resto}$
				\end{descr}
					\item[Grado 2]:\\
				\begin{descr}{0}
					\item[$f(x,y)=$]$f(x_0,y_0)+\nabla f(x_0,y_0)\scalare(x-x_0,y-y_0)+$ $+\frac{1}{2}\left[D^2f(x_0,y_0)\scalare
					\left(	\begin{array}{c}
						x-x_0\\
						y-y_0
					\end{array}\right)\right]
					\scalare 
					\left(\begin{array}{c}
						x-x_0\\
						y-y_0
					\end{array}\right)+$
					$+\underbrace{(\norma{(x-x_0,y-y_0)}^2)}_\text{Resto}$
				\end{descr}
				\item[\underline{OSS:}] (Sviluppo di Taylor vicino ad un punto critico) \\ 
					Dato $(x_0,y_0)$ punto critico, $f$ vicino a $(x_0,y_0)$ è della forma $f(x_0+h,y_0+k)$ con $h,k$ piccoli. Supponiamo inoltre che $D^2 f(x_0,y_0)=
					\begin{pmatrix}
					 	\lambda_1 & 0 \\
					 	0 & \lambda_2
					\end{pmatrix}$
					diagonale, allora, poiché il vettore delle derivate parziali è $(0,0)$ (dalla definizione di punto critico):
				$\begin{cases}
					\text{se } \lambda_1,\lambda_2>0 \implies (x_0,y_0)\text{ è un MIN. LOC.}\\
					\text{se } \lambda_1,\lambda_2<0 \implies (x_0,y_0)\text{ è un MAX. LOC.}\\
					\text{se } \lambda_1 \text{ e }\lambda_2 \text{ sono discordi} \implies (x_0,y_0)\text{ è un punto di SELLA}
				\end{cases}$
			\end{descr}

		\end{myParagraph}

%PARAGRAFO 11
		\begin{myParagraph}{FUNZIONI SCALARI IN $\bm{n}$ VARIABILI}
Prendo una funzione del tipo $f:\R^n\to\R$ allora:
			\begin{descr}{0}
				\item[Dominio] $\rightarrow$ $D:=\{\underline{x}\in\R^n|f(\underline{x})\in\R\}$;
				\item[Intorni] $\rightarrow$ $\{\underline{x}\in\R^n|\;\norma{\underline{x}-\underline{x_0}}<r\}$ è l'intorno sferico di $\underline{x_0}$;
				\item[Continuità] $\rightarrow$ \quad $\underline{n}^{(n)}\to\underline{x_0}\implies\modulo{f(\underline{n}^{(n)})-f(\underline{x_0})}\to0$?;
				\item[Gradiente] $\rightarrow$ $\nabla f(\underline{x})=\left( \frac{\partial f}{\partial x_1}(\underline{x}),...,\frac{\partial f}{\partial x_n}(\underline{x}) \right)$ vettore di $\R^n$;
				\item[Matrice Essiana] $\rightarrow$ $D^2f(\underline{x})=\text{matr. }n\times, n$,\\
				$(D^2f)_{ij}(\underline{x})=\frac{\partial^2 f}{\partial x_i\partial y_i}(\underline{x})$ (se $f\in C^2\implies D^2$è simmetrica);
				\item[Taylor] $\rightarrow$ continua a valere.
			\end{descr}
		\end{myParagraph}
		
%PARAGRAFO 12
		\begin{myParagraph}{FUNZIONI VETTORIALI IN $\bm{n}$ VARIABILI}
Prendo una funzione del tipo $F:\R^n\to\R^m$ allora possiamo considerare $F$ come avente $m$ componenti, cioè:
			\begin{tcenter}
				$F(\underline{x})=(f_1(\underline{x}),...,f_m(\underline{x})) \text{ con } f_i:\R^n\to\R$.
			\end{tcenter}
Ognuna di queste componenti ha $n$ derivate parziali. Possiamo quindi definire la Matrice Jacobiana delle derivate parziali:
			\begin{tcenter}
				$DF(\underline{x})=
				\underbrace{\begin{pmatrix}
					\frac{\partial f_1}{\partial x_1} & \dots & \frac{\partial f_1}{\partial x_n} \\
					\vdots & & \vdots \\
					\frac{\partial f_m}{\partial x_1} & \dots & \frac{\partial f_m}{\partial x_n} \\
				\end{pmatrix}}_{m\times n}$
				, con $(DF)_{ij}=\frac{\partial f_i}{\partial x_i}$.
			\end{tcenter}
		\end{myParagraph}

%PARAGRAFO 13
		\begin{myParagraph}{DERIVAZIONE DI FUNZIONI COMPOSTE}
Date 2 funzioni del tipo $f:\R^2\to\R$ e $\gamma:\R\to\R^2$, allora la derivata della loro composizione $f \circ \gamma:\R\to\R$ è:
			\begin{tcenter}
				$\frac{d}{dt}(f \circ \gamma)(t)=\underbrace{\nabla f(\gamma(t))}_{\text{vett. di }\R^2}\scalare\underbrace{\dot{\gamma}(t)}_{\text{vett. di }\R^2}=$ \\$=\frac{\partial f}{\partial x}(\gamma(t))\cdot\frac{dx}{dt}(t)+\frac{\partial f}{\partial y}(\gamma(t))\cdot\frac{dy}{dt}(t)$
			\end{tcenter}
con $\gamma$ derivabile in $t$ e $f$ differenziabile in $\gamma(t)$. \\
\bo{Generalizzazione Vettore Gradiente:}\\
Date due funzioni del tipo $f:\R^n\to\R^m$ e $g:\R^m\to\R$, allora il gradiente della loro composizione è: 
			\begin{descr}{1}
				\item[Def. Estesa:] $\nabla(g\circ f)$ è un vettore riga di $\R^n$. In particolare la $i$-esima componente di tale vettore è:\\
					\begin{tcenter}
						$\frac{\partial}{\partial x_i}(g \circ f)=\sum_j=1^n \frac{\partial g}{\partial y_j}(f(x))\cdot\frac{\partial f_j}{\partial x_i}(x)$;
					\end{tcenter}
				\item[Def. Matriciale:] $\underbrace{\nabla(g\circ f)}_{1\times n}=\underbrace{\nabla g}_{1\times m} \scalare \underbrace{Df}_{m \times n}$.
				\begin{descr}{0}
					\item[\underline{OSS:}] $\nabla(g\circ f)(\underline{x})=\nabla g(f(\underline{x}))\scalare Df(\underline{x})$ (come la derivata classica).
				\end{descr}
			\end{descr}
\bo{Generalizzazione Matrice Jacobiana:}\\
Date due funzioni vettoriali del tipo $f:\R^n\to\R^m$ e $g:\R^m\to\R^k$ la loro composizione sarà $g\circ f:R^n\to\R^k$. Allora la matrice Jacobiana delle derivate parziali è data da:
			\begin{tcenter}
				$\underbrace{D(g\circ f)(\underline{x})}_{k\times n}=\underbrace{\nabla g(f(\underline{x}))}_{k\times m}\scalare \underbrace{Df(\underline{x})}_{m\times n}$
			\end{tcenter}
		\end{myParagraph}

		\begin{myParagraph}{EQUAZIONI DELLE CURVE}
			\begin{descr}{0}
				\item[Parabola] $\rightarrow$ $p:=ax^2+bx+c=0$ con $a$ concavità con centro in $(-\frac{b}{2a},-\frac{\Delta}{4a})$.
				\item[Circonferenza] $\rightarrow$ $c:=(x-\alpha)^2+(y-\beta)^2=r^2$ con $\alpha,\beta$ la distanza dall'origine del centro, $r$ il raggio.
				\item[Ellisse] $\rightarrow$ $e:=\frac{(x-\alpha)^2}{a^2}+\frac{(y-\beta)^2}{b^2}=1$ con $\alpha,\beta$ la distanza dall'origine del centro, $a$ è il semiasse su $x$ e $b$ è il semiasse su $y$.
				\item[Iperbole] $\rightarrow$ $i:=\frac{(x-\alpha)^2}{a^2}-\frac{(y-\beta)^2}{b^2}=1$ con asintoti di equazione $y=-\frac{bx}{a}$ e $y=\frac{bx}{a}$ (in caso $\alpha,\beta\neq0$ si aggiungono nella relativa variabile cioè si pone $x=x-\alpha$ e $y=y-\beta$).1
			\end{descr}
		\end{myParagraph}

%PARAGRAFO 15
		\begin{myParagraph}{PARAMETRIZZAZIONE DELLE CURVE}
%Una curva è una funzione $\gamma:I\subset\R\to\R^n$. Uso $t\in I$ come variabile (tempo). Allora:
%			\begin{descr}{0}
%				\item[Vettore Posizione] $\rightarrow$ $\gamma(t)=(\gamma_1(t),...,\gamma_n(t))$ con $\gamma_i\in \R$;
%				\item[Vettore Velocità] $\rightarrow$ se $\exists \dot{\gamma_i}, \dot{\gamma_i}:I\to\R^n$ allora: 
%				\begin{tcenter}
%					$\dot{\gamma}(t)=(\dot{\gamma_1}(t),...,\dot{\gamma_n}(t))$;
%				\end{tcenter}
%				\item[Vettore Accelerazione] $\rightarrow$ se $\exists \ddot{\gamma_i}, \ddot{\gamma_i}:I\to\R^n$ allora:
%				\begin{tcenter}
%					$\ddot{\gamma}(t)=(\ddot{\gamma_1}(t),...,\ddot{\gamma_n}(t))$;
%				\end{tcenter}
%				\item[Sostegno] $\rightarrow$ $\Gamma=\gamma(I)$ (è un' "immagine" della curva). Curve diverse possono avere lo stesso sostegno.
%				\item[Curva Semplice] $\rightarrow$ Data $\gamma:[a,b]\to\R^n$, questa si dice Semplice se $\gamma(t_1)\neq \gamma(t_2)\quad \forall t_1\neq t_2$ con almeno uno dei $t_i\in(a,b)$.
%				\begin{descr}{0}
%					\item[\underline{OSS:}] $\gamma$ è semplice $\SSE\gamma_{|(a,b)}$
%è iniettiva.
%				\end{descr}
%				\item[Curva Chiusa] $\rightarrow$ Una curva $\gamma$ si dice chiusa se $\gamma(a)=\gamma(b)$.
%				\item[Curva Regolare] $\rightarrow$ Una curva $\gamma$ si dice regolare se $\in C^1((a,b);\R)$ (cioè ogni componente è di classe $C^1$) e se $\dot{\gamma}\neq\underline{0}\quad \forall t\in (a,b)$.
%				\begin{descr}{0}
%					\item[\underline{OSS:}] $\exists$ sempre il vettore velocità ed è $\neq 0 \implies \exists \;\tau(t)=\frac{\dot{\gamma}(t)}{\norma{\dot{\gamma}(t)}}$ (viene detto Versore Tangente).
%				\end{descr}
%			\end{descr}
			\begin{descr}{0}
				\item[Segmenti] $\rightarrow$ $s:=[t, mt+q]$ con $t\in[t_1,t_2]$\;(se il segmento è verticale diventa $[mt+q, t]$).
				\item[Parabola] $\rightarrow$ $p:=[t, at^2+bt+c]$ con $t\in[t_1,t_2]$
 				\item[Circonferenza] $\rightarrow$ $c:=[\alpha+r \cos(t), \beta+r \sin(t)]$ con $\alpha,\beta$ la distanza dall'origine del centro, $r$ il raggio e $t\in[0,2\pi]$.
				\item[Ellisse] $\rightarrow$ $e:=[\alpha+ a \cos(t),\beta + b \sin(t)]$ con $\alpha,\beta$ la distanza dall'origine del centro, $a$ è il semiasse su $x$ e $b$ è il semiasse su $y$ e $t\in[0,2\pi]$.
				\item[Iperbole] $\rightarrow$ $i:=[\cosh(t),\sinh(t)]$ dove $\cosh(t)=\frac{e^t+e^{-t}}{2}$ e $\sinh(t)=\frac{e^t-e^{-t}}{2}$.
			\end{descr}
 		\end{myParagraph}

%PARAGRAFO 16
		\begin{myParagraph}{FORMULE TRIGONOMETRICHE}
			\begin{descr}{0}
				\item[Relazione Fondamentale]:\\
					$\cos^2(\alpha)+\sin^2(\alpha)=1$.
				\item[Formule di Duplicazione]:\\
					$\sin(2\alpha)=2\sin(\alpha)\cos(\alpha)$;\\
					$\cos(2\alpha)=\cos^2(\alpha)-\sin^2(\alpha)$;\\
					$\tan(2\alpha)=\frac{2\tan(\alpha)}{1-\tan^2(\alpha)}$.
				\item[Formule di Bisezione]:\\
					$\sin(\frac{\alpha}{2})=\pm\sqrt{\frac{1-\cos(\alpha)}{2}}$;\\
					$\cos(\frac{\alpha}{2})=\pm\sqrt{\frac{1+\cos(\alpha)}{2}}$;\\
					$\tan(\frac{\alpha}{2})=\pm\sqrt{\frac{1-\cos(\alpha)}{1+\cos(\alpha)}}$;\\
			\end{descr}
		\end{myParagraph}

%PARAGRAFO 17
		\begin{myParagraph}{DERIVATE DI FUNZIONI}
			\begin{center}
				\begin{tabular}{ |c|c| } 
\hline
\bo{Funzione} & \bo{Derivata}  \\ 
\hline

$\tan(x)$ & $\frac{1}{\cos^2(x)}$ \\ 
\hline
$\cot(x)$ & $-\frac{1}{\sin^2(x)}$  \\ 
\hline
$\arcsin(x)$ & $\frac{1}{\sqrt{1-x^2}}$ \\
\hline
$\arccos(x)$ & $-\frac{1}{\sqrt{1-x^2}}$ \\
\hline
$\arctan(x)$ & $\frac{1}{1-x^2}$ \\
\hline
				\end{tabular}
			\end{center}
		\end{myParagraph}
	
	
	
	\end{formulario}
\end{document}